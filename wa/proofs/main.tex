\documentclass[fleqn,10pt]{olplainarticle}

% \renewcommand{\labelenumi}{(\alph{enumi})}
% \renewcommand{\labelenumii}{\roman{enumii}}

% Connect to reference file
\addbibresource{references.bib}

% Prepare title and abstract
\title{
    \normalsize{
        City College of San Francisco
    }\\
    \large{
        Discrete Mathematics Peer Review Assignment
    }\\
    \vspace{0.5ex}
    \Huge{
        Asymptotic Behavior
    }
    \vspace{0.5ex}
}
\author{} % Leave blank for the assignment
\begin{abstract}
    The purpose of this assignment is to help you apply the concepts learned in the lessons to prepare you to analyze essential features of recursive and iterative algorithms. Also, you are preparing to write proofs using symbolic logic, Boolean algebra, and mathematical induction. These are Student Learning Outcomes for the course. Additionally, the peer review process gives you a structured way to learn how to critique the validity of another individual's argument, and it provides me with evidence of regular student-to-student contact in the course.  
    
\end{abstract}

\begin{document}

    % Generate Title
    \maketitle
    \flushbottom
    
    % Remove page number of first page
    \thispagestyle{empty} 
    
    \textit{Do not include your name on this document, so the peer-review process can be as anonymous as possible. As long as you upload this assignment through your Canvas account, then your work will be connected to your name for grading purposes. I have set the peer review parameters in Canvas to be anonymous.}

            
    % Problem
    \section*{Problem 14.20 - Modified}
        Identify each of the following states as true or false. Proof that your response is correct.
        
        \begin{enumerate}
            \item $n^2 \sim n^2 + n$
            \item $3^n = O\left(2^n\right)$
            \item $n = \Theta\left(\frac{3n^3}{n^2 - 1}\right)$
        \end{enumerate}
        
        % Preliminary Notes
        \subsection*{Required Preliminary Notes}
        
            This problem seems to just require a definition verification.
            
        % Problem Response
        \subsection*{Required Problem Response}
            \begin{enumerate}
                \item Since
                
                    \begin{align*}
                        \limsup_{n \to \infty} \frac{n^2}{n^2 + n} = \lim_{n \to \infty} \frac{1}{1 + 1/n} = 1,
                    \end{align*}
                    
                     Definition 14.4.2 confirms that $n^2 \sim n^2 + n$ is true. 
                     
                 \item Since
                
                    \begin{align*}
                        \limsup_{n \to \infty} \frac{|3^n|}{2^n} = \lim_{n \to \infty} \frac{3^n}{2^n} = \lim_{n \to \infty} \left(\frac{3}{2}\right)^n = \infty,
                    \end{align*}
                    then $3^n = O\left(2^n\right)$ is false according to Definition 14.7.5.
                    
                \item Since 
                    \begin{align*}
                        \limsup_{n \to \infty} \frac{|n|}{\frac{3n^3}{n^2 -1}} = \lim_{n \to \infty} \frac{n(n^2 + 1)}{3n^3} = \lim_{n \to \infty} \frac{1 + \frac{1}{n^2}}{3} = \frac{1}{3}
                    \end{align*}
                    
                    and 
                    \begin{align*}
                        \limsup_{n \to \infty} \frac{\left|\frac{3n^3}{n^2 -1}\right|}{n} = \lim_{n \to \infty} \frac{3n^3}{n^3 - n} = \lim_{n \to \infty} \frac{3}{1 - \frac{1}{n^2}} = 3
                    \end{align*}
                    
                    $n = O\left(\frac{3n^3}{n^2 - 1}\right)$ and $\frac{3n^3}{n^2 - 1} = O(n)$. This means that $n = \Theta\left(\frac{3n^3}{n^2 - 1}\right)$ is true by Definition 14.7.13.
                
            \end{enumerate}
            
            
            
    % Problem
    \section*{Problem 14.34}
        
        \begin{falseclaim}
            \begin{equation}
                2^n = O(1). \label{assertion}
            \end{equation}
        \end{falseclaim}
        
        \begin{falseproof}
            The proof is by induction on $n$ where the induction hypothesis $P(n)$ is the assertion (\ref{assertion}).
            
            \begin{itemize}
                \item[] \textbf{Base Case}: $P(0)$ holds trivially.
                \item[] \textbf{Inductive Step}: We may assume $P(n)$, so there is a constant $c > 0$ such that $2^n \leq c \cdot 1$. Therefore 
                
                \begin{equation*}
                    2^{n+1} = 2\cdot 2^n \leq (2c) \cdot 1,
                \end{equation*}
                
                which implies that $2^{n+1} = O(1)$. That is, $P(n+1)$ holds, which completes the proof of the inductive step.
            \end{itemize}
            
            We conclude by induction that $2^n = O(1)$ for all $n$. That is, the exponential function is bounded by a constant. QED.
        \end{falseproof}
        
        Identify a major mistake in the above \textbf{False Proof}.

        % Preliminary Notes
        \subsection*{Required Preliminary Notes}
        
            The issue with this proof seems to extended from the section in the textbook called Constant Confusion.
            
        % Problem Response
        \subsection*{Required Problem Response}
            
            The induction proof correctly shows that for every value of $n$, $2^n = O(1)$. The major mistake is that this is not what represents the claim (\ref{assertion}). There is not a fixed value for $n$, so $2^n$ is not constant. In other words, it is true that all constants as functions are $O(1)$, but the assertion (\ref{assertion}) is concerned with if the non-constant function $2^n$ is $O(1)$. To address the claim, the proof would need to consider this variability when defining the constants $c$ and $n_0$ in Definition 14.7.9.
        
           
    % Attribution and References 
    \vspace{4 em}
    
    \section*{Attribution}
        This peer-reviewed assignment contains problems from the textbook Mathematics for Computer Science \cite{MIT}. These problems are either directly from that source or modified with additional or alternate instructions as allowed by the Creative Commons Attribution-ShareAlike 3.0 license. 
        
    % Display the References
    \printbibliography

\end{document}
